% Options for packages loaded elsewhere
\PassOptionsToPackage{unicode}{hyperref}
\PassOptionsToPackage{hyphens}{url}
%
\documentclass[
  ignorenonframetext,
]{beamer}
\title{Basic commands}
\author{Karen Cristine Goncalves}
\date{2023-01-26}

\usepackage{pgfpages}
\setbeamertemplate{caption}[numbered]
\setbeamertemplate{caption label separator}{: }
\setbeamercolor{caption name}{fg=normal text.fg}
\beamertemplatenavigationsymbolsempty
% Prevent slide breaks in the middle of a paragraph
\widowpenalties 1 10000
\raggedbottom
\setbeamertemplate{part page}{
  \centering
  \begin{beamercolorbox}[sep=16pt,center]{part title}
    \usebeamerfont{part title}\insertpart\par
  \end{beamercolorbox}
}
\setbeamertemplate{section page}{
  \centering
  \begin{beamercolorbox}[sep=12pt,center]{part title}
    \usebeamerfont{section title}\insertsection\par
  \end{beamercolorbox}
}
\setbeamertemplate{subsection page}{
  \centering
  \begin{beamercolorbox}[sep=8pt,center]{part title}
    \usebeamerfont{subsection title}\insertsubsection\par
  \end{beamercolorbox}
}
\AtBeginPart{
  \frame{\partpage}
}
\AtBeginSection{
  \ifbibliography
  \else
    \frame{\sectionpage}
  \fi
}
\AtBeginSubsection{
  \frame{\subsectionpage}
}
\usepackage{amsmath,amssymb}
\usepackage{lmodern}
\usepackage{iftex}
\ifPDFTeX
  \usepackage[T1]{fontenc}
  \usepackage[utf8]{inputenc}
  \usepackage{textcomp} % provide euro and other symbols
\else % if luatex or xetex
  \usepackage{unicode-math}
  \defaultfontfeatures{Scale=MatchLowercase}
  \defaultfontfeatures[\rmfamily]{Ligatures=TeX,Scale=1}
\fi
% Use upquote if available, for straight quotes in verbatim environments
\IfFileExists{upquote.sty}{\usepackage{upquote}}{}
\IfFileExists{microtype.sty}{% use microtype if available
  \usepackage[]{microtype}
  \UseMicrotypeSet[protrusion]{basicmath} % disable protrusion for tt fonts
}{}
\makeatletter
\@ifundefined{KOMAClassName}{% if non-KOMA class
  \IfFileExists{parskip.sty}{%
    \usepackage{parskip}
  }{% else
    \setlength{\parindent}{0pt}
    \setlength{\parskip}{6pt plus 2pt minus 1pt}}
}{% if KOMA class
  \KOMAoptions{parskip=half}}
\makeatother
\usepackage{xcolor}
\IfFileExists{xurl.sty}{\usepackage{xurl}}{} % add URL line breaks if available
\IfFileExists{bookmark.sty}{\usepackage{bookmark}}{\usepackage{hyperref}}
\hypersetup{
  pdftitle={Basic commands},
  pdfauthor={Karen Cristine Goncalves},
  hidelinks,
  pdfcreator={LaTeX via pandoc}}
\urlstyle{same} % disable monospaced font for URLs
\newif\ifbibliography
\setlength{\emergencystretch}{3em} % prevent overfull lines
\providecommand{\tightlist}{%
  \setlength{\itemsep}{0pt}\setlength{\parskip}{0pt}}
\setcounter{secnumdepth}{-\maxdimen} % remove section numbering
\ifLuaTeX
  \usepackage{selnolig}  % disable illegal ligatures
\fi

\begin{document}
\frame{\titlepage}

\begin{frame}
\end{frame}

\begin{frame}[fragile]{Basics}
\protect\hypertarget{slide1}{}
\begin{itemize}[<+->]
\tightlist
\item
  Commands are separated by a new line (Enter, Return) or a semicolon
  (;)
\item
  The first word in a command is what you are asking the computer to do
  (a function)
\item
  Spaces are used to separate file names, commands, etc.
\item
  Some commands allow you to customize their output with options

  \begin{itemize}[<+->]
  \tightlist
  \item
    These are single letters, a word, or several words preceded by
    \texttt{-} or \texttt{-\/-}
  \end{itemize}
\item
  You can get help for a command by using the help option (either
  \texttt{-h} , \texttt{-\/-help} ) or the command \texttt{man} (for
  manual)
\end{itemize}

\begin{block}{}
\protect\hypertarget{section}{}
\begin{verbatim}
ls --help
man ls
\end{verbatim}
\end{block}

\begin{block}{}
\protect\hypertarget{section-1}{}
\end{block}
\end{frame}

\begin{frame}[fragile]{About spaces and file names}
\protect\hypertarget{slide2}{}
\begin{itemize}[<+->]
\tightlist
\item
  Put name in quotes if it has spaces (test code below)
\end{itemize}

\begin{block}{}
\protect\hypertarget{section-2}{}
\begin{verbatim}
touch Programming Class.txt
ls

touch "Programming Class.txt"
ls
\end{verbatim}
\end{block}

\begin{block}{}
\protect\hypertarget{section-3}{}
\begin{itemize}[<+->]
\tightlist
\item
  Quotes are NORMALLY (not always) interchangeable (test code below)
\end{itemize}
\end{block}

\begin{block}{}
\protect\hypertarget{section-4}{}
\begin{verbatim}
echo "This is good"
echo 'This is good'
\end{verbatim}
\end{block}

\begin{block}{}
\protect\hypertarget{section-5}{}
\begin{itemize}[<+->]
\tightlist
\item
  Do not start with one quote type and end with another
\end{itemize}
\end{block}
\end{frame}

\begin{frame}[fragile]{Shortcuts}
\protect\hypertarget{slide3}{}
\begin{itemize}[<+->]
\tightlist
\item
  {\texttt{\textasciitilde{}} }or {\texttt{\$HOME} }: your home folder
  (can be defined by the user)
\item
  {\texttt{.}} : the folder that you are currently in
\item
  {\texttt{..}}: the folder that contains the one you are currently in
\item
  {\texttt{Ctrl+C}}: cancel a command
\item
  In MobaxTerm, find and modify useful shortcuts by clicking on Settings
  -\textgreater{} Keyboard shortcuts
\end{itemize}
\end{frame}

\begin{frame}[fragile]{Shortcuts}
\protect\hypertarget{slide4}{}
\begin{itemize}[<+->]
\tightlist
\item
  {\texttt{whoami}}

  \begin{itemize}[<+->]
  \tightlist
  \item
    prints your username (if saved in the computer)
  \end{itemize}
\item
  Use \texttt{tab} to complete words
\item
  In a current command or in a text file, move the cursor faster by
  using \texttt{Ctrl+Arrow} (right or left arrow in a command line, in a
  text file up and down arrow too) (this works everywhere!!!)
\end{itemize}
\end{frame}

\begin{frame}[fragile]{Basic commands}
\protect\hypertarget{slide5}{}
\begin{itemize}[<+->]
\tightlist
\item
  {\texttt{cd}}

  \begin{itemize}[<+->]
  \tightlist
  \item
    acronym for ``change directory'' (directory = folder)
  \item
    If used alone, opens your home folder
  \item
    The name of the folder to which you want to go comes after
    \texttt{cd}

    \begin{itemize}[<+->]
    \tightlist
    \item
      \texttt{cd} , \texttt{cd\ \textasciitilde{}} and
      \texttt{cd\ \$HOME} are synonyms
    \end{itemize}
  \item
    \texttt{cd\ -} - takes you to the previous folder
  \end{itemize}
\end{itemize}
\end{frame}

\begin{frame}[fragile]{Basic commands}
\protect\hypertarget{slide6}{}
\begin{itemize}[<+->]
\tightlist
\item
  {\texttt{pwd}}

  \begin{itemize}[<+->]
  \tightlist
  \item
    acronym for ``print working directory'' (directory = folder)
  \item
    equivalent to the R function \texttt{getwd()} or python's
    \texttt{os.getcwd()}
  \item
    prints the full path to your current folder
  \item
    A full path always starts from the root (/)
  \end{itemize}
\end{itemize}
\end{frame}

\begin{frame}[fragile]{Basic commands}
\protect\hypertarget{slide7}{}
\begin{itemize}[<+->]
\tightlist
\item
  {\texttt{ls}}

  \begin{itemize}[<+->]
  \tightlist
  \item
    lists the contents of your current folder
  \item
    Check \protect\hyperlink{2}{slide 2} where we used this command
  \end{itemize}
\item
  Use {\texttt{\textgreater{}}} after a command to save the output
\end{itemize}

\begin{block}{}
\protect\hypertarget{section-6}{}
\begin{verbatim}
pwd > myFolder
cat myFolder
\end{verbatim}
\end{block}

\begin{block}{}
\protect\hypertarget{section-7}{}
\begin{itemize}[<+->]
\tightlist
\item
  Use {\texttt{\textgreater{}\textgreater{}}} to add the current output
  to a previous file
\end{itemize}
\end{block}

\begin{block}{}
\protect\hypertarget{section-8}{}
\begin{verbatim}
ls >> myFolder
cat myFolder
\end{verbatim}
\end{block}

\begin{block}{}
\protect\hypertarget{section-9}{}
\end{block}
\end{frame}

\begin{frame}[fragile]{Managing text files}
\protect\hypertarget{slide8}{}
\begin{itemize}[<+->]
\tightlist
\item
  {\texttt{cat}} : prints the contents of the file to the screen (check
  \protect\hyperlink{5}{slide 5})

  \begin{itemize}[<+->]
  \tightlist
  \item
    Do not use it with files are that not text (images, pdfs, compressed
    files) or is too big
  \item
    If several files names are put after the command, one file is
    printed followed by the next (conCATenation)
  \end{itemize}
\item
  {\texttt{head}} and {\texttt{tail}}: prints to the screen the
  first/last 10 lines of the file
\end{itemize}
\end{frame}

\begin{frame}[fragile]{Managing text files}
\protect\hypertarget{slide9}{}
\begin{itemize}[<+->]
\tightlist
\item
  {\texttt{less}} : opens the file as ``read-only''

  \begin{itemize}[<+->]
  \tightlist
  \item
    Search for a word in a file inside less by typing ``/'' followed by
    the word
  \item
    To close less, press Q
  \end{itemize}
\item
  {\texttt{more}} : opens the file as ``read-only'', when the file is
  closed, prints it to the screen
\end{itemize}
\end{frame}

\begin{frame}[fragile]{Managing text files}
\protect\hypertarget{slide10}{}
\begin{itemize}[<+->]
\tightlist
\item
  {\texttt{nano}} : open a text file to edit it.
\item
  {\texttt{grep}} : searches for a word/phrase in the file and prints
  the lines that match

  \begin{itemize}[<+->]
  \tightlist
  \item
    Can search for several phrases (one per line) in a file by using the
    option \texttt{-f}
  \item
    If you don't care about the upper/lower case, use the option
    \texttt{-i} or \texttt{-\/-ignore-case}
  \end{itemize}
\end{itemize}
\end{frame}

\begin{frame}[fragile]{Managing text files}
\protect\hypertarget{slide11}{}
\begin{block}{}
\protect\hypertarget{section-10}{}
\# {Search for lines that have the word programming in the file myFolder
created in \protect\hyperlink{7}{slide7}}

\texttt{grep\ "Programming"\ myFolder}
\end{block}

\begin{block}{}
\protect\hypertarget{section-11}{}
\begin{itemize}[<+->]
\tightlist
\item
  {\texttt{wc}} : word count. Counts the number of characters, words and
  lines in a file
\item
  {\texttt{echo}} : repeats the text that follows it (check
  \protect\hyperlink{2}{slide 2})
\end{itemize}
\end{block}
\end{frame}

\begin{frame}[fragile]{Exercise}
\protect\hypertarget{slide12}{}
\begin{itemize}[<+->]
\tightlist
\item
  Save a fasta file into your home folder with the name myFasta.fa
\item
  Use \texttt{grep} to find all the lines with sequence IDs.

  \begin{itemize}[<+->]
  \tightlist
  \item
    Note - put the word or phrase you will search for inside \texttt{""}
  \end{itemize}
\end{itemize}
\end{frame}

\begin{frame}[fragile]{Exercise - Solution}
\protect\hypertarget{slide13}{}
\begin{itemize}[<+->]
\tightlist
\item
  Save a fasta file into your home folder with the name myFasta.fa
\item
  Use \texttt{grep} to find all the lines with sequence IDs.

  \begin{itemize}[<+->]
  \tightlist
  \item
    Note - put the word or phrase you will search for inside \texttt{""}
  \end{itemize}
\end{itemize}

\begin{block}{}
\protect\hypertarget{section-12}{}
\#{ In all fasta files, the sequence ID line is indicated by the symbol
\textgreater, so we just need to look for it}

\texttt{grep\ "\textgreater{}"\ myFasta.fa}
\end{block}

\begin{block}{}
\protect\hypertarget{section-13}{}
\begin{itemize}[<+->]
\tightlist
\item
  Normally, if you just search for one word, the quotes are not needed,
  by in this case, the symbol \texttt{"\textgreater{}"} could also mean
  ``send the output to'', which would replace the myFasta.fa file
\end{itemize}
\end{block}
\end{frame}

\begin{frame}[fragile]{Managing files}
\protect\hypertarget{slide14}{}
\begin{itemize}[<+->]
\tightlist
\item
  {\texttt{cp}} : acronym for copy

  \begin{itemize}[<+->]
  \tightlist
  \item
    {\texttt{cp\ file\ file2}} : creates a copy of the file ``file'' and
    saves it in the file ``file2''
  \item
    {\texttt{cp\ file\ folder}} : creates a copy of the file ``file''
    and saves it with the same name in the folder ``folder''
  \item
    {\texttt{cp\ file\ file2\ folder}}: both files ``file'' and
    ``file2'' are copied into the folder ``folder'' with the same names
  \item
    {\texttt{cp\ folder\ folder2\ -r}} : the option -r allows the copy
    of the entire folder.

    \begin{itemize}[<+->]
    \tightlist
    \item
      If folder2 doesn't exist, it will be created to hold the same
      files as ``folder''
    \item
      If folder2 exists, a copy of ``folder'' will be created inside of
      folder2
    \end{itemize}
  \end{itemize}
\end{itemize}
\end{frame}

\begin{frame}[fragile]{Managing files}
\protect\hypertarget{slide15}{}
\begin{itemize}[<+->]
\tightlist
\item
  {\texttt{mv}} : acronym for move. Move file from one place to another

  \begin{itemize}[<+->]
  \tightlist
  \item
    {\texttt{mv\ file\ file2}} : renames ``file'' as ``file2''
  \item
    {\texttt{mv\ file\ folder}} : moves ``file'' into the folder
    ``folder''
  \item
    {\texttt{mv\ file\ file2\ folder}}: both files ``file'' and
    ``file2'' are moved into the folder ``folder'' with the same names
  \item
    {\texttt{mv\ folder\ folder2}} :

    \begin{itemize}[<+->]
    \tightlist
    \item
      If folder2 doesn't exist, it is the same as renaming ``folder'' as
      ``folder2''
    \item
      If folder2 exists, ``folder'' is moved to that folder
    \end{itemize}
  \end{itemize}
\end{itemize}
\end{frame}

\begin{frame}[fragile]{Managing files}
\protect\hypertarget{slide16}{}
\begin{itemize}[<+->]
\tightlist
\item
  {\texttt{rm}} : acronym for remove. Deletes files. \textbf{They are
  PERMANENTELY deleted, there is no trash bin here!!!!!!!}

  \begin{itemize}[<+->]
  \tightlist
  \item
    {\texttt{rm\ file\ file2}} : deletes both files
  \item
    {\texttt{rm\ file\ file2\ -i}} : asks the user if they really want
    to delete each file, if \texttt{y} is pressed, the file is deleted
    (\texttt{-i} for interactive)
  \end{itemize}
\end{itemize}
\end{frame}

\begin{frame}[fragile]{Managing folders}
\protect\hypertarget{slide17}{}
\begin{itemize}[<+->]
\tightlist
\item
  {\texttt{mkdir}} : acronym for ``make directory''. Creates new folders
  with the specified names.

  \begin{itemize}[<+->]
  \tightlist
  \item
    Gives an error if something with the same name already exists in the
    current folder.
  \item
    If many names are given (separated by spaces, creates all names
    folders)
  \end{itemize}
\item
  {\texttt{rmdir}} : acronym for ``remove directory''. Deletes
  \textbf{\emph{empty}} folders.

  \begin{itemize}[<+->]
  \tightlist
  \item
    If the folder is not empty, gives an error.
  \item
    If many names are given (separated by spaces, deletes all names
    folders \textbf{if they are empty}
  \end{itemize}
\item
  {\texttt{rm\ -r\ folder}} : as in \texttt{cp}, the option \texttt{-r}
  allows the \texttt{rm} to work with a folder. It deletes everything in
  the folder, then deletes the folder itself
\end{itemize}
\end{frame}

\begin{frame}{Resources for help}
\protect\hypertarget{resources-for-help}{}
\href{https://github.com/KarenGoncalves/Amaryllidaceae_database/wiki/Glossary-of-commands}{Glossary
of commands}

\href{https://eriqande.github.io/eca-bioinf-handbook/}{Book on basic
unix commands}
\end{frame}

\end{document}
